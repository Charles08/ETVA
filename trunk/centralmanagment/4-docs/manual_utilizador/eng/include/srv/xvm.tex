\subsection{XVM}

O serviço de monitorização em cada máquina real chama-se xvmd e é executado automaticamente pelo supervise.

Para verificar o estado do processo de monitorização:

\begin{Verbatim}[commandchars=\\\{\}]
# svstat /service/xvm
/service/xvm: up (pid 2283) 6530 seconds
\end{Verbatim}

Para arrancar ou reiniciar o processo de monitorização:

\begin{Verbatim}[commandchars=\\\{\}]
# svc -u /service/xvm
# svc -t /service/xvm
\end{Verbatim}

A script de controlo e manipulação das máquinas virtuais chama-se xvm. Detalhes sobre todos os comandos da script virão mais tarde, deixa-se agora apenas alguns casos úteis em emergência:

arrancar uma máquina virtual:

\begin{Verbatim}[commandchars=\\\{\}]
# xvm --start tmp1
\end{Verbatim}

terminar uma máquina virtual:

\begin{Verbatim}[commandchars=\\\{\}]
# xvm --stop tmp1
\end{Verbatim}

terminar forçadamente uma máquina virtual:

\begin{Verbatim}[commandchars=\\\{\}]
# xvm --stop --destroy tmp1
\end{Verbatim}

cancelar a definição de arranque automático:

\begin{Verbatim}[commandchars=\\\{\}]
# xvm --set --auto no tmp1
\end{Verbatim}

ligar-se à consola de uma máquina virtual:

\begin{Verbatim}[commandchars=\\\{\}]
# xvm --console tmp1
\end{Verbatim}

listagem das máquinas virtuais e seu estado:

\begin{Verbatim}[commandchars=\\\{\}]
# xvm -l
n=app1                       r=xen13                      a=xen13
...
n=tmp1                        r=                           a=xen12
n=ws2                        r=xen14                      a=xen14
\end{Verbatim}

criar máquina virtual para-virtualizada:

\begin{Verbatim}[commandchars=\\\{\}]
# /usr/sbin/xvm --add openerp --mem 2048 --max-mem 8064 --vcpus 2 \\
     --vlan 100 --vbd /dev/mapper/openerp
\end{Verbatim}

instanciar um servidor virtual windows e iniciá-lo:

\begin{Verbatim}[commandchars=\\\{\}]
# /usr/sbin/xvm --add windowsxp1 --hvm --mem 2048 --max-mem 8064 \\
     --vcpus 1 --vlan 100 --vbd /dev/mapper/windowsxp1
# xvm --start windowsxp1  ;  xvm --cdrom windowsxp1 \\
     /root/windowsxp\_32bit.iso ; xvm --reboot windowsxp1 -y ; \\
     sleep 2 ; xvm --console windowsxp1
\end{Verbatim}

modificar os parâmetros de uma máquina virtual

\begin{Verbatim}[commandchars=\\\{\}]
# /usr/sbin/xvm --set openerp --mem 1024
# /usr/sbin/xvm --dump openerp > /xen/openerp/config.xm
\end{Verbatim}

migrar todas as máquinas virtuais a correr na real xen1 para a real xen2:

\begin{Verbatim}[commandchars=\\\{\}]
# xvm -l | while read l; do eval $l; [ "$r" = "xen1" ] &&
  xvm --migrate $n xen2; done
\end{Verbatim}

detalhes sobre máquinas virtuais:

\begin{Verbatim}[commandchars=\\\{\}]
# xvm -l -v tmp1
tmp1:
  dn:            etXenName=tmp1,ou=XenGuests,dc=xvm
  memory:        7500
  max memory:    7168
  vcpus:         1
  auto start:    xen1
  running in:    xen1
  vlan:            96 (mac: 00:16:3e:a1:a1:a8)
  vbd:           hda1 -> /dev/vgsata/root-tmp1
  vbd:           hda2 -> /dev/vgsas/swap-tmp1
\end{Verbatim}

Para os casos excepcionais em que a script xvm se encontre inoperacional, apresentam-se de seguida os comandos para manipulação directa das máquinas virtuais.

Para iniciar uma máquina virtual devemos fazer:

\begin{Verbatim}[commandchars=\\\{\}]
# xm create /xen/tmp1/config.xm
\end{Verbatim}

É extremamente importante garantir que esta máquina virtual não está a correr noutro servidor, correndo o risco de corromper dados.

Para ver que máquinas virtuais estão a correr no servidor em questão pode-se efectuar:

\begin{Verbatim}[commandchars=\\\{\}]
# xm list
Name                                      ID Mem(MiB) VCPUs State   Time(s)
Domain-0                                   0      510     4 r-----   6024.1
dns1                                       3      195     1 -b----    514.4
dns2                                       4      195     1 -b----    281.3
fw1                                        1       95     1 -b----    129.2
fw2                                        2       95     1 -b----    159.5
\end{Verbatim}

Existe também o comando `xm top' que permite visualizar, muito ao estilo do comando top, as máquinas virtuais em execução e a sua utilização.

Para migrar uma máquina virtual entre dois nós é necessário que a máquina virtual não tenha qualquer partição em discos específicas à máquina real. Devem estar todos nem VGs (Volume Groups) partilhados. Pode então ser efectuado o seguinte comando:

\begin{Verbatim}[commandchars=\\\{\}]
# xm migrate --live tmp1 xen3
\end{Verbatim}

Se o live migrate não funcionar deverá ser efectuado o migrate normal.

A responsabilidade de arranque das máquinas virtuais de um servidor é do daemon xvmd que as iniciará no seu arranque.

É extremamente importante que não seja arrancada a mesma máquina virtual em
máquinas reais diferentes, uma vez que isso causaria corrupção do filesystem.
Por essa razão, o sistema de arranque automático do xen (/etc/xen/auto e
/etc/init.d/xendomains) não deve ser usado.

