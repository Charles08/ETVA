\subsection{Sistema de \emph{Failover}}

O serviço de \emph{Failover} garante a redundancia dos servicos configurados e a elevada disponibilidade de si próprio garantindo a redundância dele próprio.

O ficheiro de configuração deste serviço é:

\begin{Verbatim}[commandchars=\\\{\}]
/etc/sysconfig/ha/lvs.cf
\end{Verbatim}

Como exemplo podemos ver o seguinte:

Para iniciar a configuração de um novo serviço de balanceamento colocamos a seguinte directiva:

\begin{verbatim}
failover nome_do_serviço {
\end{verbatim}

A opção seguinte apenas nos diz se este novo serviço vai estar activo ou não (neste caso iria estar activo):

\begin{verbatim}
     active = 1
\end{verbatim}

Os seguintes parâmetros indicam o IP, o interface e a máscara de rede onde este serviço vai ficar atribuído:

\begin{verbatim}
     address = 10.10.10.143 eth0:143
     pmask = 255.255.255.0
\end{verbatim}

Os parâmetros seguintes indicam qual a porta que se deverá verificar o estado do servidor e o tempo que devemos aguardar por uma resposta (no exemplo abaixo, verificamos a porta de \emph{ssh} do ip de balanceamento.

\begin{verbatim}
	port = 22
	timeout = 10
\end{verbatim}

O \emph{start\_cmd} e \emph{stop\_cmd}, identificam os comandos que queremos executar para iniciar o serviço ou para-lo dependendo da verificação a cima explicada.

\begin{verbatim}
	start_cmd = "/etc/scripts/lvs-up-net.sh"
	stop_cmd = "/etc/scripts/lvs-down-net.sh"
\end{verbatim}

Para terminar a configuração apenas colocamos o caracter:

\begin{verbatim}
}
\end{verbatim}

O comando para controlar este serviço (iniciar/parar) é o seguinte:

\begin{Verbatim}[commandchars=\\\{\}]
# /etc/rc.d/init.d/pulse start
# /etc/rc.d/init.d/pulse stop
\end{Verbatim}

