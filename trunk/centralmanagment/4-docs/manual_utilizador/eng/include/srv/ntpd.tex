\subsection{NTPD}

O \emph{NTP}\footnote{Network Time Protocol} é um protocolo desenvolvido sobre IP para permitir a sincronização dos relógios dos sistemas de uma rede. A sincronização destes é efectuada a pedido, mediante um cliente (aplicação) que questiona periodicamente um servidor, obtendo amostragens regularmente e efectuando adaptações o mais pequenas possível, espalhadas pelo tempo, de forma a não alterar significativamente o horizonte temporal.

O ficheiro de configuração deste serviço é:

\begin{Verbatim}[commandchars=\\\{\}]
/etc/ntp.conf
\end{Verbatim}

O comando para controlar este serviço (iniciar/parar) é o seguinte:

\begin{Verbatim}[commandchars=\\\{\}]
# /etc/rc.d/init.d/ntpd start
# /etc/rc.d/init.d/ntpd stop
\end{Verbatim}

