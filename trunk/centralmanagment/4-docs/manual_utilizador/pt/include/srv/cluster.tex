\subsection{Cluster}

O serviço de \emph{cluster} instalado pela Eurotux na plataforma, tem como objectivo garantir a redundância dos serviços.
Estes serviços ficam activos apenas em um dos nós do \emph{cluster}, ficando no outro nó desactivados até intervenção especifica do administrador ou por falha do nó activo.

O estado do cluster poderá ser consultado pela execução do comando:

\begin{Verbatim}[commandchars=\\\{\}]
clustat
\end{Verbatim}

A configuração poderá ser \underline{consultada} no ficheiro \emph{/etc/cluster/cluster.[conf|xml]} (Dependendo da versão do cluster).

O serviço de cluster é composto por dois serviços:

\begin{itemize}
	\item cman - cluster manager
	\item rgmanager - resource group manager
\end{itemize}

O primeiro é responsável pela gestão global do cluster, nomeadamente a nível de mecanismos de fencing\footnote{Mecanismo em que um servidor impede o acesso de outro a um recurso partilhado} e de ''locking''. O segundo necessita da funcionalidade do primeiro e implementa a gestão de serviços que o cluster vai gerir, nomeadamente arrancar e parar serviços, endereços IP, etc.

O comando para controlar este serviço (iniciar/parar) é o seguinte:

\begin{Verbatim}[commandchars=\\\{\}]
# /etc/rc.d/init.d/cman start
# /etc/rc.d/init.d/cman stop
\end{Verbatim}

No entanto o comando acima não deve ser executado sem primeiro desligar o ''rgmanager'' devido às precedências. Como tal deverá primeiro executar:


\begin{Verbatim}[commandchars=\\\{\}]
# /etc/rc.d/init.d/rgmanager start
# /etc/rc.d/init.d/rgmanager stop
\end{Verbatim}

