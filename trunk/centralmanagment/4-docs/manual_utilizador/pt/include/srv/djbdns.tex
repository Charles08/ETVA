\subsection{Serviço de DNS -- djbdns}

O servidor de \emph{DNS}\footnote{Domain Name System - Sistema de Nomes de Domínios} utilizado neste caso é o Servidor \emph{djbdns}.

O \emph{djbdns} divide-se essencialmente em componentes:
\begin{description}
	\item[dnscache:] Componente responsável por realizar queries de dns recursivas. Mantém uma cache de dns, configurável, para mais rápidas respostas. Podem ser definidos quais os servidores a utilizar, bem como associações entre que servidores utilizar mediante o domínio a questionar.
	\item[tinydns and tinydns-data:] Componente que apenas contém informação autoritativa, não contendo a informação pedida localmente. A Base de dados local indica ao tinydns informação sobre determinado domínio.
	\item[axfrdns:] Componente para efectuar respostas via \emph{TCP}\footnote{Transmission Control Protocol} (porto 53), caso em que uma resposta exceda o tamanho de 512 bytes e para transferências de zona (\emph{AXFR}\footnote{Asynchronous Full Transfer Zone}).
	\item[axfr-get:] Cliente para transferências de zonas, tipicamente para criar relações do tipo Master-Slave.
\end{description}

Os ficheiros de configuração deste serviço encontram-se em locais distintos, dependendo do componente a configurar:

\begin{Verbatim}[commandchars=\\\{\}]
/etc/dnscache
/etc/tinydns
/etc/axfrdns
\end{Verbatim}

As zonas podem ser configuradas na seguinte directoria:

\begin{Verbatim}[commandchars=\\\{\}]
/etc/tinydns/root/data
\end{Verbatim}

Tipicamente o djbdns corre os componentes separadamente em daemontools, e os comandos para controlar este serviço (iniciar/parar/ver estado) são os seguintes:

\begin{Verbatim}[commandchars=\\\{\}]
# svc -u /service/_componente_
# svc -d /service/_componente_
# svstat /service/_componente_
\end{Verbatim}

