\subsection{Asterisk}

O \emph{Asterisk} é um software, que implementa uma Central Telefônica convencial, utilizando tecnologia \textsf{VoIP}\footnote{Voz sobre Protocolo Internet(IP) é um conjunto de tecnologias, protocoloas de comunicação, que tem por objectivo permitir a comunicações de voz através de rede IP, tais como a Internet \textsf{VoIP}.}. Este software, permite voz sobre o protocolo internet em quatro Protocolos\footnote{SIP: Session initiation protocol, IAX:  Inter Asterisk Exchange protocol, MGCP: Media gateway control protocol, H.323}  


Este tipo de software, pode ser instalado e executado em varios ambientes, tais como, \emph{Linux}, \emph{BSD} e \emph{Windows}.

Informações adicionais podem ser obtidas em:\\ \begin{normalsize}\sffamily\href{http://www.asterisk.org/}{www.asterisk.org}\end{normalsize}

As várias configurações do \emph{Asterisk} podem ser encontradas na seguinte directoria:

\begin{Verbatim}[commandchars=\\\{\}]
/etc/asterisk
\end{Verbatim}

O comando para controlar este serviço (iniciar/parar) é o seguinte:

\begin{Verbatim}[commandchars=\\\{\}]
# /etc/init.d/asterisk stop
# /etc/init.d/asterisk start
\end{Verbatim}

