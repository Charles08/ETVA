\subsection{Serviço de email -- qmail-ldap}

O \emph{SMTP}\footnote{Simple Mail Transfer Protocol} é um protocolo relativamente
simples, baseado em texto simples, onde as mensagens são transferidas de um emissor
para um ou mais receptores.

Especificamente este servidor de \emph{SMTP} está integrado com o conjunto de software
disponibilizado pelo \emph{qmail}.

Os ficheiros de configuração deste serviço estão situados na seguinte directoria
(atenção que estes ficheiros são comuns para o conjunto de software disponibilizado pelo
\emph{qmail}):

\begin{Verbatim}[commandchars=\\\{\}]
/srv/qmail/control
\end{Verbatim}

O comando para controlar o serviço de smtp (iniciar/parar) é o seguinte:

\begin{Verbatim}[commandchars=\\\{\}]
# svc -u /service/qmail-smtpd
# svc -dk /service/qmail-smtpd
\end{Verbatim}

O comando para controlar o serviço de envio (iniciar/parar) é o seguinte:

\begin{Verbatim}[commandchars=\\\{\}]
# svc -u /service/qmail
# svc -dk /service/qmail
\end{Verbatim}

O comando para controlar o serviço de pop3 (iniciar/parar) é o seguinte:

\begin{Verbatim}[commandchars=\\\{\}]
# svc -u /service/qmail-pop3d
# svc -dk /service/qmail-pop3d
\end{Verbatim}

O comando para controlar o serviço de pop3s (iniciar/parar) é o seguinte:

\begin{Verbatim}[commandchars=\\\{\}]
# svc -u /service/qmail-pop3d-ssl
# svc -dk /service/qmail-pop3d-ssl
\end{Verbatim}

