\subsection{Sistema de Balanceamento}

O serviço de balanceamento garante não só o balanceamente de carga das plataformas reais, mas também a redundancia dos servicos configurados.
Para além destas funcionalidades o próprio serviço balanceamento garante a elevada disponibilidade de si próprio garantindo a redundância dele próprio.

O ficheiro de configuração deste serviço é:

\begin{Verbatim}[commandchars=\\\{\}]
/etc/sysconfig/ha/lvs.cf
\end{Verbatim}

Como exemplo podemos ver o seguinte:

Para iniciar a configuração de um novo serviço de balanceamento colocamos a seguinte directiva:

\begin{verbatim}
virtual nome_do_serviço {
\end{verbatim}

A opção seguinte apenas nos diz se este novo serviço vai estar activo ou não (neste caso iria estar activo):

\begin{verbatim}
     active = 1
\end{verbatim}

Os seguintes parâmetros indicam o IP, o interface e a máscara de rede onde este serviço vai ficar atribuído:

\begin{verbatim}
     address = 10.10.10.143 eth0:143
     pmask = 255.255.255.0
\end{verbatim}

Os parâmetros seguintes indicam qual a porta e protocolo que se vai balancear e qual o par pergunta/resposta que o(s) servidor(es) deve(m) responder caso esteja(m) a funcionar correctamente:

\begin{verbatim}
     port = http
     send = "GET / HTTP/1.0\r\n\r\n"
     expect = "HTTP"
     protocol = tcp
\end{verbatim}

A seguinte opção indica o algoritmo de balanceamento a aplicar neste serviço:

\begin{verbatim}
     scheduler = lc
\end{verbatim}

A seguinte secção descreve o servidor real (se está activo ou não e o seu peso) para onde os pedidos vão ser redireccionados. 
Esta secção deve ter o número de servidores que se desejam balancear.

\begin{verbatim}
     server server1 {
         address = 192.168.0.10
         active = 1
         weight = 1
     }
\end{verbatim}

Para terminar a configuração apenas colocamos o caracter:

\begin{verbatim}
}
\end{verbatim}

O comando para controlar este serviço (iniciar/parar) é o seguinte:

\begin{Verbatim}[commandchars=\\\{\}]
# /etc/rc.d/init.d/pulse start
# /etc/rc.d/init.d/pulse stop
\end{Verbatim}

