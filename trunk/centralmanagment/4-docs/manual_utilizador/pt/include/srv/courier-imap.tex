\subsection{Serviço de IMAP -- Courier-Imap}

O \emph{IMAP}\footnote{Internet Message Access Protocol} é um protocolo utilizado para acesso a caixas de correio.
Este protocolo tenta colmatar algumas das limitações do seu tradicional rival, o POP\footnote{Post Office Protocol}.
A principal vantagem do IMAP é manter as mensagens de correio no lado do servidor, permitindo assim que a consulta das mesmas seja feita independentemente da localização ou cliente utilizado, possibilitando ainda a criação de pastas para organizar as mensagens.

O ficheiro de configuração deste serviço é:

\begin{Verbatim}[commandchars=\\\{\}]
/usr/lib/courier-imap/etc/imapd
\end{Verbatim}

Os comandos para controlar este serviço (iniciar/parar) são os seguinte:

\begin{Verbatim}[commandchars=\\\{\}]
# /usr/lib/courier-imap/libexec/imapd.rc start
# /usr/lib/courier-imap/libexec/imapd.rc stop
\end{Verbatim}

